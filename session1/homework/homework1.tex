\documentclass[12pt]{article}
\usepackage[utf8]{inputenc}
\usepackage{amsmath, amssymb, url, float, graphicx, float}
\usepackage{fullpage}
\usepackage[small,bf]{caption}
\usepackage{versions}


\newcommand{\field}{\mathbf{F}}
\newcommand{\from}{\leftarrow}
\newcommand{\impliesn}[1]{\underset{#1}{\implies}}
\newcommand{\iffn}[1]{\underset{#1}{\iff}}
\newcommand{\impliesp}{\impliesn{p}}
\newcommand{\iffp}{\iffn{p}}
\newcommand{\impliespp}{\impliesn{p'}}
\newcommand{\iffpp}{\iffn{p'}}
\newcommand{\ex}{\mathbf{E}}

\let\eps\varepsilon
\let\emptyset\varnothing
\let\phi\varphi

\let\vec\undefined
\newcommand{\vec}{\mathbf{vec}}

\excludeversion{solution}
% \includeversion{solution}

\input defs.tex

\title{Homework 1: SP\&LA Study Group}
\author{Guille}
\date{November 2023}

\begin{document} 
\maketitle 

\section*{A general note}
None of the solutions to these exercises should be more than a few lines long.
If you find yourself writing more than half a page or so for a solution,
there's probably a simpler way! (Of course, \emph{finding} the simpler way
might not, itself, be so simple.) Problem parts beginning with an {\bf (H)} are
harder problems.

\section{Matrices}
In this problem, we'll discuss some properties of matrices. For this problem,
let's fix an $m\times n$ matrix with elements in the field $\field$, written $A
\in \field^{m\times n}$. Denote the columns of $A$ by the $m$-vectors $a_1,
\dots, a_n \in \field^m$. As a reminder, if we have an $n$-vector $x \in
\field^n$, then the matrix-vector product between $A$ and $x$, which we write
as $Ax$, results in the vector
\begin{equation}\label{eq:mat-mul}
    Ax = x_1a_1 + x_2a_2 + \dots + x_n a_n,
\end{equation}
which is a linear combination of the columns of $A$ with the scalars
equaling the entries of the vector $x$.

\paragraph{Part 1.} Show that the matrix-vector product, as defined above, is
\emph{linear}, in other words that, for any $\alpha \in \field$ and any $x \in
\field^n$,
\[
    A(\alpha x) = \alpha Ax,
\]
and for any other vector $y \in \field^n$,
\[
    A(x + y) = Ax + Ay.
\]
(We will make use of this a lot in the proofs of the paper!)

% Note the \begin{solution} ... \end{solution} environment
% along with the paragraph heading!
\begin{solution}
\paragraph{Part 1 solution.} By definition we have that the scaled
vector $\alpha x$, is simply $x$ with each entry scaled by $\alpha$.
We can write this as
\[
    (\alpha x)_1 = \alpha x_1, \quad \dots, \quad (\alpha x)_n = \alpha x_n.
\]
This means that (parsing the meaning of each equality \emph{very carefully}!)
\[
    A(\alpha x) = (\alpha x)_1a_1 + \dots + (\alpha x)_na_n = \alpha (x_1 a_1 + \dots + x_n a_n) = \alpha Ax,
\]
as required.

By definition we have that the added vector $x + y$, is simply each entry of the two n-vectors added together.
We can write this as
\[
    (x + y)_1 = x_1 + y_1, \quad \dots, \quad (x + y)_n = x_n + y_n.
\]

This means, that:
\[
    A(x + y) \stackrel{(1)}{=} (x_1 + y_1) a_1 + (x_2 + y_2) a_2 + \quad \dots + (x_n + y_n) a_n
\]
This is equal to:
\[
    (x_1 a_1 + \dots + x_n a_n) + (y_1 a_1 + \dots + y_n a_n) \stackrel{(1)}{=} Ax + Ay
\]
\end{solution}

\paragraph{(H) Part 2.} Let $f: \field^n \to \field^m$ be any linear function that maps
$n$-vectors to $m$-vectors;
\ie, the function $f$ is \emph{linear} because it satisfies
\[
    f(\alpha x) = \alpha f(x)
\]
and 
\[
    f(x + y) = f(x) + f(y),
\]
for any scalar $\alpha \in \field$ and any vectors $x, y \in \field^n$.
(Compare this with the definition above!) Show that there exists some matrix $A
\in \field^{m \times n}$ such that
\[
    f(x) = Ax.
\]
(The fact that every matrix corresponds to a linear function and every linear
function corresponds to a matrix is the reason this field is called
\emph{linear algebra}.)

\emph{Hint.} Note that every vector $x \in \field^n$ can be written
as a linear combination of the basis vectors,
\[
    e_1 = \begin{bmatrix}
        1\\0\\\vdots\\0
    \end{bmatrix},
    ~~
    e_2 = 
    \begin{bmatrix}
        0\\1\\\vdots\\0
    \end{bmatrix},
    ~~
    \dots,
    ~~
    e_n = 
    \begin{bmatrix}
        0\\0\\\vdots\\1
    \end{bmatrix}.
\]

\paragraph{Part 3.} Say we have a second linear function $h: \field^m \to
\field^k$. Let's define the new function
\[
    w(x) = h(f(x)).
\]
This new function $w$ corresponds to taking the output $f(x)$, which is an
$m$-vector, and passing it through $h$. Show that $w$ is also linear.

\begin{solution}
\paragraph{Part 3 solution.}

We know that $f$ and $h$ are linear functions, and $w: \field^n \to
\field^k$. Then for any $\alpha \in \field$ and any $x \in \field^n$ we have:

\[
    w(\alpha x) = h(f(\alpha x)) = h(\alpha f(x)) = \alpha h(f(x)) = \alpha w(x)
\]
and for any other vector $y \in \field^n$:
\[
    w(x + y) = h(f(x + y)) = h(f(x) + f(y)) = h(f(x)) + h(f(y)) = w(x) + w(y)
\]
Therefore, $w$ is also linear.

\end{solution}

\paragraph{(H) Part 4.} From part 2, we know that $w(x) = Dx$ for some matrix
$D \in \field^{k \times n}$ as it, too, is a linear function. If $f(x) = Ax$
and $h(y) = By$ for some matrix $B \in \field^{k\times m}$, then what does the
matrix $D$ correspond to in terms of $A$, $B$, or their corresponding columns?

(If you've seen linear algebra before, this question is `easy' with outside
tools, but you should not use other knowledge of linear algebra for this
question! Only the matrix-vector product from the definition
in~\eqref{eq:mat-mul} is necessary.)

\section{Vector spaces}
In this problem, we'll explore some basic vector spaces that we talked about in the
lecture.

As a reminder, we say that a set of $n$-vectors, $V \subseteq \field^n$ is a
\emph{vector space} if, for any two vectors in this set, $x, y \in V$, and any
two scalars $\alpha, \beta \in \field$, the linear combination of these two
vectors (with scalars $\alpha$, $\beta$) are also in the set $V$; \ie,
\[
    \alpha x + \beta y \in V.
\]

\paragraph{Part 1.} From the lecture, the range $\range(A)$
is defined as the set containing all possible linear combinations of the
columns of the matrix $A$. Written in set builder notation, this is
\[
    \range(A) = \{Ax \mid x \in \field^n\}.
\]
Show that the range $\range(A)$ of any matrix $A \in \field^{m\times n}$ is a
vector space. (You are welcome to use the solution to problem 1, part 1, even
if you haven't solved it.)

\paragraph{Part 2.} Using the above, the paper shows in~\S1.1 that the set of
evaluations of polynomials of degree at most $s$ on a fixed set of points is
itself a vector space. In particular, if we fix a set of points $\alpha_1,
\dots, \alpha_m \in \field$ and we define the set of vectors
\begin{equation}\label{eq:rs-mat}
    V = \{(f(\alpha_1), f(\alpha_2), \dots, f(\alpha_m)) \in \field^m \mid \text{$f$ is a polynomial of degree $\le n-1$}\},
\end{equation}
then this set $V$ is a vector space. Flesh out the proof in the paper to show
that this is indeed a vector space! (As a side note, we will make use of this
property in our proof of the security of FRI.)

\paragraph{(H) Part 3.} The paper also discusses the fact that, for every
vector space $V$, there exists a \emph{parity check matrix} $C \in
\field^{k\times m}$ such that, for some vector $x \in \field^m$, we have that
$x \in V$ if, and only if, $Cx = 0$. Write out the parity check matrix
corresponding to the vector space defined in~\eqref{eq:rs-mat}.
Two hints: first, take a look at the definition of the Vandermonde matrix
in the paper. Second, look up \emph{Lagrange interpolation}.

\paragraph{(H) Other fun.} For bonus points, show that, indeed, every vector space
has a parity check matrix. You may assume that every vector space $V$ has a
matrix $A$ such that $\range(A) = V$ and that this matrix has linearly
independent columns. (See below for a reminder of the definition of linear
independence.)

\section{The $\ell_0$ `norm'}
From before, remember that we defined the `norm' $\|y\|_0$ of a vector $y$ in a
finite field as the number of nonzero entries of the vector $y$. In math,
that is
\[
    \|y\|_0 = |\{i\mid y_i \ne 0\}|.
\]
We will show three properties we will use throughout the paper.

\paragraph{Part 1.} Show that $\|y\|_0 = 0$ if, and only if, $y = 0$. (This
is called \emph{definiteness}.)

\paragraph{Part 2.} Show that, for any two vectors $x, y \in \field^n$
the \emph{triangle inequality} holds; \ie,
\[
    \|x + y\|_0 \le \|x\|_0 + \|y\|_0.
\]

\paragraph{Part 3.} Show that the $\ell_0$ `norm' is zero-homogeneous; \ie,
\[
    \|\alpha x\|_0 = \|x\|_0,
\]
for any $\alpha \in \field$ that is nonzero, $\alpha \ne 0$. (This should be
contrasted with a usual norm over the reals or the complex numbers!)

\paragraph{Part 4.} A usual thing to do in linear algebra over the real
numbers is to take an \emph{inner product} of two vectors $x$ and $y$ with
the same number of elements, $n$. This is defined as
\[
    x^Ty = x_1y_1 + x_2y_2 + \dots + x_ny_n.
\]
The notion of \emph{orthogonality}, that is, that two vectors $x$ and $y$ have
inner product equal to zero; \ie,
\[
    x^Ty = 0,
\]
plays a big role when the vectors' elements are real numbers. In that case, if
$x$ and $y$ are orthogonal, we can say that $x$ and $y$ form a right angle,
and, more importantly, no vector is orthogonal to itself, except the all-zeros
vector. Unfortunately this notion of orthogonality is not so clear in finite
fields.

Show that there is a finite field $\field$ and a nonzero vector $x \in
\field^n$ that is orthogonal to itself; \ie, the vector $x$ satisfies $x^Tx =
0$. (You are free to choose $n$, the number of elements in the vector.) Extend
this example to show that, for any finite field $\field$, there is always a
nonzero vector $x \in \field^n$, that is orthogonal to itself. (As a hint, you
should use the fact that adding 1 to itself $|\field|$ times results in 0.)

\section{Linear independence and distance}
In this problem, we will relate the distance of a code with some basic
properties of linear independence.

From before, we define the \emph{distance} $d$ of a matrix $G \in \field^{m\times n}$
as
\[
    d = \min_{x \ne 0} \|Gx\|_0
\]
where $\|y\|_0$ denotes the number of nonzero entries of the vector $y$.
Similarly, we say that a matrix $G$ has \emph{linearly independent} columns if
its nullspace contains only the zero vector; \ie, if
\[
    \nullspace(G) = \{0\},
\]
where the nullspace $\nullspace(G)$ is defined
\[
    \nullspace(G) = \{y \in \field^n \mid Gy = 0\}.
\]
A basic fact from linear algebra (which you are free to use here) is that any
matrix $G$ with linearly independent columns has at least as many rows as it
has columns; \ie, since $G$ is an $m\times n$ matrix, then $m \ge n$.

\paragraph{Part 1.} Show that a generator matrix $G$ is linearly independent
if, and only if, its distance $d$ is positive, $d > 0$.

\paragraph{Part 2.} Given a matrix $G$ which has some distance $d > 0$, show
that we can remove any $d-1$ rows to get a new matrix $\tilde G \in
\field^{(m-d + 1)\times n}$ that also has linearly independent columns. Argue that
this means that the distance must satisfy
\[
    d \le m - n + 1.
\]
(This bound on the distance is known as the \emph{Singleton bound} in coding
theory.) Codes that achieve this bound at equality are known as
`maximum-distance separable' or MDS codes.

As a fun sidenote: if $G$ is a generator matrix for the Reed--Solomon code
(see~\S1.3.2), then we know that $d = m - n + 1$, making the Reed--Solomon code
an MDS code. In a sense, it is the highest-distance code for the chosen
dimensions (message length and block size).

\section{(Probabilistic?) Implications}
In this section, we'll explore both `traditional' logic implications
and probabilistic logic implications. As a reminder, given two statements
$P$ and $Q$, we say $P$ \emph{implies} $Q$ when
\begin{equation}\label{eq:implication}
    \neg(P \wedge \neg Q).
\end{equation}
As a second reminder, given random variables $r$ and $r'$ which are taken
from some (known) distribution and statements $P_r$ and $Q_{r'}$, each
depending on the randomness of $r$ and $r'$, then we say that
\[
    P_r \impliesp Q_{r'}
\]
whenever $\Pr(P_r \wedge \neg Q_{r'}) \le p$.


\paragraph{Part 1.} Show that if $P$ implies $Q$ and $Q$ implies $T$, then $P$
implies $T$ using the definition of implication given
above in~\eqref{eq:implication}. You will need to assume the \emph{law of
excluded middle}: either $Q$ or $\neg Q$. This is called the \emph{transitivity}
of implication.

(It is actually possible to have logic without the law of excluded middle, but
implications must be defined differently for transitivity to hold. Lucky for
us, we only deal with finite---if very large---sets so we can forget any of
this conversation ever happened.)

\paragraph{Part 2.} Prove that the probabilistic implications have a similar
transitivity property; in particular, if $P_r \impliesp Q_{r'}$ and $Q_{r'}
\impliespp T_{r''}$ then $P_r \impliesn{p + p'} T_{r''}$. (This proof is
provided in appendix~A of the paper, but you are encouraged to only peek if you
need a hint :)

\paragraph{Part 3.} A common special case will be to take $Q$ to be a
deterministic event (\ie, the statement does not depend on any randomness).
This is true in many ZK protocols as we want to ensure some logical (\ie,
deterministic) statement is true but somehow `reduce' this claim down to an
easier-to-check claim over a smaller statement with some randomness.

First, show that in this case, $P_r \impliesp Q$ is the same as the statement
\[
    \Pr(P_r) \le p
\]
when $\neg Q$. (In some way, we may think of this as: the probability of $P_r$,
given $\neg Q$, is less than $p$. If $p$ is very small, like $2^{-80}$, yet we
observe $P_r$, then it is very unlikely that $\neg Q$.)

Second, let's say we have $n$ statements, all depending on the same randomness
$r$, given by $P_r^1 \impliesp Q^1$, $P_r^2 \impliesp Q^2$, until, $P_r^n
\impliesp Q^n$. Show that
\[
    P_r^1 \wedge P_r^2 \wedge \dots \wedge P_r^n \impliesp Q^1 \wedge Q^2 \wedge \dots \wedge Q^n.
\]
(We use a special case of this fact in the proof in~\S3.1.2 in the paper.)
\emph{Hint.} This should be \emph{very} simple. If you find yourself writing more
than a few lines, then it's likely you're overthinking it!


\end{document}
