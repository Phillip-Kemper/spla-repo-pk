\documentclass{beamer}

\let\phi\varphi
\newcommand{\field}{\mathbf{F}}
\newcommand{\impliesn}[1]{\underset{#1}{\implies}}
\newcommand{\iffn}[1]{\underset{#1}{\iff}}
\newcommand{\impliesp}{\impliesn{p}}
\newcommand{\iffp}{\iffn{p}}
\newcommand{\impliespp}{\impliesn{p'}}
\newcommand{\iffpp}{\iffn{p'}}
\newcommand{\ex}{\mathbf{E}}


\mode<presentation>
{
\usetheme{default}
}
\setbeamertemplate{navigation symbols}{}
\usecolortheme[rgb={0.13,0.28,0.59}]{structure}
\setbeamertemplate{itemize subitem}{--}
\setbeamertemplate{frametitle} {
        \begin{center}
          {\large\bf \insertframetitle}
        \end{center}
}

\newcommand\footlineon{
  \setbeamertemplate{footline} {
    \begin{beamercolorbox}[ht=2.5ex,dp=1.125ex,leftskip=.8cm,rightskip=.6cm]{structure}
      \footnotesize \insertsection
      \hfill
      {\insertframenumber}
    \end{beamercolorbox}
    \vskip 0.45cm
  }
}
\footlineon

\AtBeginSection[]
{       
        \begin{frame}<beamer> 
                \frametitle{Outline} 
                \tableofcontents[currentsection]
        \end{frame}
}

% \usepackage{subcaption}
\usepackage{amsmath,amssymb,amsthm}
\usepackage{xcolor}
\usepackage{graphicx}
\usepackage{tikz}
\usetikzlibrary{matrix, positioning}

\makeatletter
\def\mathcolor#1#{\@mathcolor{#1}}
\def\@mathcolor#1#2#3{%
  \protect\leavevmode
  \begingroup
    \color#1{#2}#3%
  \endgroup
}
\makeatother

\title{Linear Algebra and Probabilistic Implications}
\author{Guillermo Angeris \and Alex Evans}
\date{Session 1, SPLA Study Club}

\newcommand{\ones}{\mathbf 1}
\newcommand{\reals}{{\mbox{\bf R}}}
\newcommand{\integers}{{\mbox{\bf Z}}}
\newcommand{\symm}{{\mbox{\bf S}}}  % symmetric matrices

\newcommand{\nullspace}{{\mathcal N}}
\newcommand{\range}{{\mathcal R}}
\newcommand{\Rank}{\mathop{\bf Rank}}
\newcommand{\tr}{\mathop{\bf tr}}
\newcommand{\diag}{\mathop{\bf diag}}
\newcommand{\card}{\mathop{\bf card}}
\newcommand{\rank}{\mathop{\bf rank}}
\newcommand{\conv}{\mathop{\bf conv}}
\newcommand{\prox}{\mathbf{prox}}

\newcommand{\Expect}{\mathop{\bf E{}}}
\newcommand{\Prob}{\mathop{\bf Prob}}
\newcommand{\Co}{{\mathop {\bf Co}}} % convex hull
\newcommand{\dist}{\mathop{\bf dist{}}}
%\newcommand{\argmin}{\mathop{\rm argmin}}
%\newcommand{\argmax}{\mathop{\rm argmax}}
\DeclareMathOperator*{\argmin}{argmin}
\DeclareMathOperator*{\argmax}{argmax}
\newcommand{\epi}{\mathop{\bf epi}} % epigraph
\newcommand{\Vol}{\mathop{\bf vol}}
\newcommand{\dom}{\mathop{\bf dom}} % domain
\newcommand{\intr}{\mathop{\bf int}}
\newcommand{\sign}{\mathop{\bf sign}}

\newcommand{\cf}{{\it cf.}}
\newcommand{\eg}{{\it e.g.}}
\newcommand{\ie}{{\it i.e.}}
\newcommand{\etc}{{\it etc.}}

\newcommand{\BEAS}{\begin{eqnarray*}}
\newcommand{\EEAS}{\end{eqnarray*}}
\newcommand{\BEA}{\begin{eqnarray}}
\newcommand{\EEA}{\end{eqnarray}}
\newcommand{\BEQ}{\begin{equation}}
\newcommand{\EEQ}{\end{equation}}
\newcommand{\BIT}{\begin{itemize}}
\newcommand{\EIT}{\end{itemize}}

\begin{document}
	\begin{frame}
		\titlepage
	\end{frame}
    
    \section{High level ideas}
    \begin{frame}
        \frametitle{At a high level}
        \begin{itemize}\itemsep=12pt
            \item Take a bunch of concepts from succinct proofs (ZK)
            \item Reduce them to linear algebra
            \item (and a bit of error correcting codes)
        \end{itemize}
    \end{frame}

    \begin{frame}
        \frametitle{At a high level (cont.)}
        \begin{itemize}\itemsep=12pt
            \item Introduce succinct (!) notation
            \item Relax traditional logic to `probabilistic' versions
            \item Get `proof-carrying' protocols at the end!
            \item Show a (weak) bound on the soundness of FRI
            \pause
            \item To do this, we have to eat some veggies first...
        \end{itemize}
    \end{frame}
    \begin{frame}
        \frametitle{Why do this work?}
        \begin{itemize}\itemsep=12pt
            \pause
            \item We start with something we know exists
            \item Try to (a) find minimal requirements for it to work
            \item And (b) try to use this to clean up exposition
            \pause
            \item Why?
        \end{itemize}
    \end{frame}

    \begin{frame}
        \frametitle{Why do this work? (Cont.)}
        \begin{itemize}\itemsep=12pt
            \item Often, removing requirements helps understanding
            \item Allows generalizations and new discoveries
            \item Lets us divide a protocol into its constituent parts
            \pause
            \item For more on this check out ep.\ 294 with Kobi and Anna :)
        \end{itemize}
    \end{frame}

    
    \begin{frame}
        \frametitle{But in reality}
        \begin{itemize}\itemsep=12pt
            \pause
            \item We do it because it's fun :)
        \end{itemize}
    \end{frame}
	
    \section{Linear algebra}
    % -- Notation
    \begin{frame}
        \frametitle{Vectors}
        \begin{itemize}\itemsep=12pt
            \item An $n$-\emph{vector} is an ordered collection of $n$ elements
            \item Example: a $3$-vector $x$
            \[
                x = \begin{bmatrix}
                    3\\
                    5\\
                    1
                \end{bmatrix}
            \]
            \item We will sometimes write this as $x = (3, 5, 1)$
            \pause
            \item Can \emph{index} this collection: $x_1=3$, $x_2 = 5$, $x_3=1$
        \end{itemize}
    \end{frame}
    \begin{frame}
        \frametitle{Operations on vectors}
        \begin{itemize}\itemsep=12pt
            \item We can \emph{scale} vectors by a \emph{scalar}
            \item Example: $x=(3,5,1)$ scaled by $2$
            \[
                2x = 2\begin{bmatrix}
                    3\\
                    5\\
                    1
                \end{bmatrix}
                = \begin{bmatrix}
                    6\\
                    10\\
                    2
                \end{bmatrix}
            \]
            \vspace{-2em}
            \pause
            \item We can also add vectors too
            \[
                x+\begin{bmatrix}
                    1\\
                    2\\
                    3
                    \end{bmatrix}
                = \begin{bmatrix}
                    3\\
                    5\\
                    1
                \end{bmatrix}
                + \begin{bmatrix}
                    1\\
                    2\\
                    3
                    \end{bmatrix}
                = \begin{bmatrix}
                    4\\
                    7\\
                    4
                \end{bmatrix}
            \]
        \end{itemize}
    \end{frame}
    \begin{frame}
        \frametitle{Linear combinations}
        \begin{itemize}\itemsep=12pt
            \item Of course, we can do both at the same time
            \item Given vectors $x$, $y$, and $z$, we can take
            \[
                x + 2y + 3z
            \]
            \item Called a \emph{linear combination} of vectors
        \end{itemize}
    \end{frame}
    \begin{frame}
        \frametitle{Scalar and vector notation}
        \begin{itemize}\itemsep=12pt
            \item Vectors will almost always be \emph{lowercase Roman} letters
            \item Such as $x$, $y$, $z$, ...
            \item Scalars will almost always be \emph{lowercase Greek} letters
            \item For example, $\alpha$, $\beta$, $\gamma$, ...
        \end{itemize}
    \end{frame}

    \begin{frame}
        \frametitle{Linear combinations}
        \begin{itemize}\itemsep=12pt
            \item We are often interested in linear combinations of vectors
            \item For this we need, (a) a collection of vectors and (b) a collection of scalars
            \pause
            \item From before: (b) is just a vector
            \pause
            \item But (a) is what we call a \emph{matrix}
        \end{itemize}
    \end{frame}

    \begin{frame}
        \frametitle{Matrices}
        \begin{itemize}\itemsep=12pt
            \item An $m \times n$ matrix is an ordered list (of length $n$) of $m$-vectors
            \pause
            \item We can take a linear combination of the $n$ available $m$-vectors
            \pause
            \item If a vector $z=(4, 8, 2)$ contains the scalars
            \item And $A$ is a $m \times 3$ matrix, then
            \[
                Az = 4a_1 + 8a_2 + 2a_3
            \]
            where $a_i$ is the $i$th column of $A$
            \pause
            \item Very compact notation!
        \end{itemize}
    \end{frame}

    \begin{frame}
        \frametitle{Example of a matrix}
        \begin{itemize}\itemsep=12pt
            \item Useful to write matrices as `rectangles of numbers'
            \item Example matrix $A$ of dimensions $3\times 2$
            \[
                A = \begin{bmatrix}
                    3 & 1\\
                    5 & 2\\
                    1 & 3
                \end{bmatrix}
            \]
            \item Matrices will almost always be \emph{uppercase Roman} letters
            \item Such as $A$, $B$, $C$, ...
        \end{itemize}
    \end{frame}

    \begin{frame}
        \frametitle{Finite fields (a quick aside)}
        \begin{itemize}\itemsep=12pt
            \item The numbers in the vectors (and matrices) have to `exist somewhere'
            \item We usually assume these numbers lie in a \emph{finite field} $\field$
            \item An $n$-vector $x$ from a finite field $\field$ is written
            \[
                x \in \field^n
            \]
            \item An $m\times n$ matrix $A$ with elements in $\field$ is written
            \[
                A \in \field^{m\times n}
            \]
        \end{itemize}
    \end{frame}

    \begin{frame}
        \frametitle{Vector spaces}
        \begin{itemize}\itemsep=12pt
            \item The natural sets where linear algebra works are \emph{vector spaces}
            \pause
            \item A set $V \subseteq \field^n$ is a vector space if it is \emph{closed} under linear combinations
            \item If we take $x, y \in V$ then
            \[
                \alpha x + \beta y \in V
            \]
            for any $\alpha, \beta \in \field$
        \end{itemize}
    \end{frame}

    \begin{frame}
        \frametitle{Vector space examples}
        \begin{itemize}\itemsep=12pt
            \item The simplest vector space: $V = \{0\}$
            \pause
            \item Second simplest: $V = \field^n$
            \pause
            \item Third (?) simplest, for fixed $x \in \field^n$: $V = \{\alpha x \mid \alpha \in \field\}$
        \end{itemize}
    \end{frame}

    \begin{frame}
        \frametitle{Range and nullspace}
        \begin{itemize}\itemsep=12pt
            \item The \emph{range} of a matrix $A \in \field^{m\times n}$ is defined as
            \[
                \range(A) = \{Ax \mid x \in \field^n\}
            \]
            is a vector space (see homework)
            \pause
            \item Similarly, the \emph{nullspace} of the matrix $A$
            \[
                \nullspace(A) = \{y \in \field^n \mid Ay = 0\}
            \]
            is also a (very different!) vector space
        \end{itemize}
    \end{frame}

    
    \begin{frame}
        \frametitle{Codes}
        \begin{itemize}\itemsep=12pt
            \item We will call a matrix $G \in \field^{m \times n}$ a (linear) \emph{error correcting code}
            \item The matrix $G$ takes in an $n$-vector and spits out an $m$-vector
            \item Given a \emph{message} $x \in \field^n$ then
            \[
                y = Gx
            \]
            is an \emph{encoding} of the message $x$
        \end{itemize}
    \end{frame}

    \begin{frame}
        \frametitle{Codes (examples)}
        \begin{itemize}\itemsep=12pt
            \item Some classic codes
            \pause
            \item The identity code
            \[
                G = I
            \]
            such that $Gx = x$
            \pause
            \item The repeated code!
            \[
                Gx = \begin{bmatrix} x\\x\\\vdots\\x \end{bmatrix}
            \]
        \end{itemize}
    \end{frame}

    \begin{frame}
        \frametitle{Codes (examples, cont.)}
        \begin{itemize}\itemsep=12pt
            \item More examples!
            \pause
            \item The Hadamard code $G \in \field^{m\times n}$: every possible
                $n$-tuple is a row of $G$
            \pause
            \item The Reed--Solomon code $G \in \field^{m \times n}$ (see homework!)
        \end{itemize}
    \end{frame}

    \begin{frame}
        \frametitle{Distance}
        \begin{itemize}\itemsep=12pt
            \item We will only mainly use one definition which is
                the \emph{distance} of a code
            \item Defined as
            \[
                d = \min_{x \ne 0} \|Gx\|_0
            \]
            \item Here, $\|y\|_0$ is the number of nonzero elements in $y$
        \end{itemize}
    \end{frame}

    \begin{frame}
        \frametitle{Distances of some codes}
        \begin{itemize}\itemsep=12pt
            \item The identity code $Gx = x$ has $d = 1$
            \pause
            \item The repeated code has $d = k$, where $k$ is the number of repetitions
            \pause
            \item The Hadamard code has distance $d = |\field|^n - |\field|^{n-1}$
            \pause
            \item The Reed--Solomon code has distance $d = m - n + 1$ (see homework!)
        \end{itemize}
    \end{frame}

    \section{Probabilistic implications}
    \begin{frame}
        \frametitle{Probabilistic implications}
        \begin{itemize}\itemsep=12pt
            \item We'll first start with `normal' logic
            \item Then expand to a probabilistic logic framework
        \end{itemize}
    \end{frame}

    \begin{frame}
        \frametitle{`Traditional' logic}
        \begin{itemize}\itemsep=12pt
            \item In classical logic, we have statements, say $P$ and $Q$
            \pause
            \item They are either true or false
            \pause
            \item In some cases, we can say basic
                things, such as
            \[
                P \wedge Q
            \]
            (read: $P$ and $Q$)

            \item Note: statements in math are {\bf assertions}!
        \end{itemize}
    \end{frame}

    \begin{frame}
        \frametitle{Implications}
        \begin{itemize}\itemsep=12pt
            \item Other possible statements include
            \item $P$ implies $Q$ (\ie, some statement implies another)
            \pause
            \item This is the same as saying
            \[
                \neg (P \wedge \neg Q)
            \]
            (Consequence: if $P$ implies $Q$, and $Q$ implies $T$, then $P$ implies $T$)
            \item See homework!
        \end{itemize}
    \end{frame}

    \begin{frame}
        \frametitle{Probabilistic? Implications}
        \begin{itemize}\itemsep=12pt
            \item At a high level:
            \item Zero knowledge proofs deal with implications \emph{with some error}
            \pause
            \item How do we codify this idea in notation?
            \pause
            \item By relaxing implications to \emph{probabilistic implications}
        \end{itemize}
    \end{frame}

    \begin{frame}
        \frametitle{Probabilistic implications}
        \begin{itemize}\itemsep=12pt
            \item In zero knowledge protocols, statements depend on \emph{randomness}
            \item \ie, we have some statement $P_r$
            \item Which depends on a randomly drawn $r$ (from some distribution)
        \end{itemize}
    \end{frame}

    \begin{frame}
        \frametitle{Probabilistic implications (cont.)}
        \begin{itemize}\itemsep=12pt
            \item Traditional implication: $P$ implies $Q$ if
            \[
                \neg(P \wedge \neg Q)
            \]
            \item Equivalently: implication doesn't hold if $P \wedge \neg Q$
            \pause
            \item A relaxation is: if $P_r$ and $Q_{r'}$ depend on randomness $r$ and $r'$
            then
            \[
                \Pr(P_r \wedge \neg Q_{r'}) \le p
            \]
            where $p$ is probability of error
            \item Recover original definition whenever $p=0$
        \end{itemize}
    \end{frame}

    \begin{frame}
        \frametitle{Consequences}
        \begin{itemize}\itemsep=12pt
            \item Define convenient notation
            \[
                P_r \impliesp Q_{r'}
            \]
            for $\Pr(P_r \wedge \neg Q_{r'}) \le p$
            \item Then we can chain implications!
            \[
                P_r \impliesp Q_{r'} ~~ \text{and} ~~ Q_{r'} \impliespp T_{r''}
            \]
            implies that
            \[
                P_r \impliesn{p + p'} T_{r''}
            \]
        \end{itemize}
    \end{frame}

    \begin{frame}
        \frametitle{Consequences (cont.)}
        \begin{itemize}\itemsep=12pt
            \item We can also take contrapositives
            \item Given
            \[
                P_r \impliesp Q_{r'}
            \]
            \item This is the same as
            \[
                \neg Q_{r'} \impliesp \neg P_r
            \]
            \item And a few others (see homework!)
        \end{itemize}
    \end{frame}


    \begin{frame}
        \frametitle{As a side note}
        \begin{itemize}\itemsep=12pt
            \item One can create a basic logical language
            \item Acts much like a syntax with rules for (probabilistic) proofs
            \item Except it can also spit out probability of failure
            \item An open project would be to formalize this!
        \end{itemize}
    \end{frame}

    \section{Where to from here?}

    \begin{frame}
        \frametitle{Continuing}
        \begin{itemize}\itemsep=12pt
            \item From here, we have all the basics!
            \pause
            \item Next up: rewriting some (many?) basic tools of succinct proofs
            \pause
            \item We'll finally start saying real things!
            \pause
            \item Homework will be released after lecture and Q\&A
        \end{itemize}
    \end{frame}
    % Eaten our vegetables, and really delved in. Next lecture we'll start doing some fun stuff
    % and generalize some of the open knowledge!


\end{document}
